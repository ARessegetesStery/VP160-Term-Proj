\documentclass{book}
\usepackage{booktabs}
\usepackage{geometry}
\usepackage{enumerate}
\usepackage{setspace}
\usepackage{amsthm}
\usepackage{float}
\usepackage{color, soul}
\usepackage{amsmath,amssymb}
\usepackage{amstext}
\usepackage{makecell}
\usepackage{subfigure}
\usepackage{graphicx}
\usepackage{array}
\begin{document}
{\huge 4.1 Projectile motion with air drag}

{\Large 1.}
\vspace{0.01\textheight}
According to the formular for acceleration:
{\"r}$=-g-\frac{k}{m}v$

$\begin{cases}
    v_{x_{0}}= & v_{0}cos\alpha \\

    v_{y_{0}}= & v_{0}sin\alpha
  \end{cases}$
\quad $
  \begin{cases}
    a_{x_{i}}= & -\frac{k}{m}v_{x_{i}}   \\

    a_{y_{i}}= & -g-\frac{k}{m}v_{y_{i}}
  \end{cases}
$

\vspace{0.01\textheight}
Using Euler's method:
\vspace{0.01\textheight}

\quad $
  \begin{cases}
    v_{x_{i+1}}= & v_{x_{i}}+a_{x_{i}}\Delta t = v_{x_{i}}(1-\frac{k}{m}\Delta t)           \\

    v_{y_{i+1}}= & v_{y_{i}}+a_{y_{i}}\Delta t = v_{y_{i}}(1-\frac{k}{m}\Delta t)-g\Delta t
  \end{cases}
$

\vspace{0.01\textheight}

\quad $
  \begin{cases}
    x_{i+1}= & x_{i}+v_{x_{i}}\Delta t \\

    y_{i+1}= & y_{i}+v_{y_{i}}\Delta t
  \end{cases}
$

\vspace{0.01\textheight}
{\Large 2.}
The initial conditions are set as: $x_{0}=y_{0}=0, \alpha =\frac{\pi}{6}$.

The following graph in Figure 1 shows that the numerical result is not sensitive to the choice of the step {$\Delta t$}

\begin{figure}[H]
  \centering
  \includegraphics[width=11cm,height=6cm]{project4.1.2(4).png}
  \caption{4.1.2(1). Examine the sensitivity of the numerical result to the choice of the step $\Delta t$}
\end{figure}

\quad The analytical result is given as:

\quad $\begin{cases}
    x(t)= & v_{0}cos\alpha \cdot \frac{m}{k}(1-e^{-\frac{k}{m}t})                          \\
    y(t)= & (v_{0}sin\alpha + \frac{mg}{k})\frac{m}{k}(1-e^{-\frac{k}{m}t})-\frac{mg}{k}t)
  \end{cases}$

With the analytical formulas and numerical values, three trajectories for different steps of time are plotted in as belows:

The red lines are the trajectories plotted using Euler's method, and the blue lines are trajectories plotted using analytical formulas. It can be observed from the graphs in Figure 2 that, the analytical formulas and numerical formulas reach quite close results, and smaller steps of time can lead to better approximation to the result.
\begin{figure}[H]
  \centering
  \subfigure[$\Delta t=0.08s$]{
    \includegraphics[scale=0.32]{project4.1.2(1).png}}
  \hspace{0.05in}
  \subfigure[$\Delta t=0.107s$]{
    \includegraphics[scale=0.32]{project4.1.2(2).png}}
  \hspace{0.05in}
  \subfigure[$\Delta t=0.16s$]{
    \includegraphics[scale=0.32]{project4.1.2(3).png}}
  \caption{4.1.2(2). Plot the trajectories for three different values of $\Delta t$}
\end{figure}

{\Large 3.(a)}
\begin{figure}[H]
  \centering
  \includegraphics[scale=0.3]{project4.1.3(a).png}
  \caption{4.1.3(a). Plot five trajectories for different initial angles}
\end{figure}
Choose the step to be 0.08s.
The initial speed of the projectile is fixed to be 90m/s, and the problem is solved numerically for five different values of angles.
  {The five angles are selected to be $\frac{\pi}{4},\frac{\pi}{5},\frac{\pi}{6},\frac{\pi}{7},\frac{\pi}{8}.$}, and the graph is shown in Figure 3.

The shape of the trajectories are quite similar as all of them increasely steadily and then precipitate sharply.
It can be observed the figure that, the larger the initial angle to the horizontal is, the higher the maximum height is, and the shorter the range is.

\newpage
{\Large 3.(b)}
\begin{figure}[H]
  \centering
  \includegraphics[width=9cm,height=6.3cm]{project4.1.3(b).png}
  \caption{4.1.3(b). Plot the time dependence of the speed of the particle for different initial angles}
\end{figure}
It can be observed from the graph in Figure 4 that, the speed of the particle drops quickly at the first several seconds, and then the speed become stable.

\vspace{0.03\textheight}
{\Large 4.(a)}
The initial speed of the projectile is fixed to be 90m/s, and the initial angle is set to be $\pi/6$. Five different values for k is selected as 0.5,0.75,1,1.25 and 1.5.

From the graph in Figure 5, it can be observed that the larger the drag coefficient k is, the lower the maximum height is, and the shorter the range of the projectile is.
\begin{figure}[H]
  \centering
  \includegraphics[scale=0.3]{project4.1.4(a).png}
  \caption{4.1.4(a). Plot five trajectories for different drag coefficient k}
\end{figure}
\vspace{0.01\textheight}
{\Large 4.(b)}

\begin{figure}[H]
  \centering
  \includegraphics[width=9cm,height=6cm]{project4.1.4(b).png}
  \caption{4.1.4(b). Plot the time dependence of the speed of the particle for different drag coefficient k}
\end{figure}

It can be observed from the graph in Figure 6 that, the speed of the particle drops quickly at the first several seconds, and then the speed become stable.
A tendency can also be clearly noticed that the larger the drag coefficient k is, the smaller the speed of the particle is at any instance of time.

\vspace{0.01\textheight}
{\Large 5.} According to the formula for acceleration: {\"r}=$-g-\beta |v|v=-g-\frac{b}{m}|v|v$

$\begin{cases}
    v_{x_{0}}= & v_{0}cos\alpha \\

    v_{y_{0}}= & v_{0}sin\alpha
  \end{cases}$
\quad $
  \begin{cases}
    a_{x_{i}}= & -\frac{b}{m} \sqrt{v_{x_{i}}^{2}+v_{y_{i}}^{2}}v_{x_{i}}   \\

    a_{y_{i}}= & -g-\frac{b}{m} \sqrt{v_{x_{i}}^{2}+v_{y_{i}}^{2}}v_{y_{i}}
  \end{cases}
$

\vspace{0.01\textheight}
Using Euler's method:
\vspace{0.01\textheight}

$\begin{cases}
    v_{x_{i+1}}= & v_{x_{i}}(1-\frac{b}{m}\sqrt{v_{x_{i}}^{2}+v_{y_{i}}^{2}}\Delta t)            \\

    v_{y_{i+1}}= & -g\Delta t+v_{y_{i}}(1-\frac{b}{m}\sqrt{v_{x_{i}}^{2}+v_{y_{i}}^{2}}\Delta t)
  \end{cases}$

\vspace{0.006\textheight}
$
  \begin{cases}
    x_{i+1}= & x_{i}+v_{x_{i}}\Delta t \\

    y_{i+1}= & y_{i}+v_{y_{i}}\Delta t
  \end{cases}
$

\newpage
{\Large 6.}
\begin{figure}[H]
  \centering
  \includegraphics[scale=0.5]{project4.1.6.png}
  \caption{4.1.6. Examine the sensitivity of the numerical result to the choice of the step $\Delta t$}
\end{figure}
Three trajectories are plotted according to three different values of {$\Delta t$}, which are respectively 0.16s 0.08s,and 0.016s.
It can be ovserved from the graph in Figure 7 that, the numerical result is sensitive to the choice of the step {$\Delta t$}.

{\Large 7.}

\end{document}